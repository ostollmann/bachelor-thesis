\chapter{Discussion}
\label{discussion}

Using the results presented in chapter~\ref{results} this chapter aims to discuss the positives and negatives of the approach described in the thesis.

As described in chapter~\ref{motivation}, the extraction of Markov Processes from Simulink models offers a number of advantages for the optimization of complex dynamic systems. The ability to automatically transfer system dynamics from a Simulink representation to that of a Markov Process would allow scientists and engineers to develop non-myopic controllers that take uncertainty into account, without the difficulty of having to implement these models by manually producing multi-dimensional condition transition probability, conditional observation probability and reward matrices. The test model used in chapter~\ref{results}, a third-order Butterworth filter, provides a good example. Even though this system is relatively simple, with only a single input and a single output, a manual implementation of its dynamics as a Markov Decision Process would have entailed defining a $636\times636\times10$ tensor by hand. This example underscores the usefulness of an automatic transformation. A successful transformation means that engineers can concentrate on modelling and leave the error-prone process of manually converting models to transition, observation and reward probabilities to an automated extraction system, whilst still taking advantages of the Markov Process's favourable optimization features.

For a prototype implementation, the extraction worked relatively well for the Butterworth filter. The responses to both disturbance signals more or less compare to the original model's reponses (see figures~\ref{bw_val_const_real_mean} and~\ref{bw_val_const_real_mean}). Nonetheless, deficits are visible. The much greater response variance of the MDP's response, suggests, that the number of probabilistic extracton simulations (see sections~\ref{sec:probabilisticsimulation} and~\ref{sec:resultsextraction}) should have been increased in order to correctly identify the real random behaviour of the model. Such a requirement does however lead to higher computational extraction cost. The easiest approach to this problem is to take advantage of the extractor's parallelization features to extract on machines with a high number of CPU-cores. Further study is required to identify the number of samplings necessary to truly \textit{extract} the system's random behaviour.

The results do not show an accuracy loss due to discretization. It is, in fact impossible to notice response differences caused by this loss of information (compare figures~\ref{bw_val_const_real_box} and~\ref{bw_val_const_state_box} or figures~\ref{bw_val_step_real_box} and~\ref{bw_val_step_state_box}). It must however be noted, that this accuracy will decrease if a rougher state space granularity is chosen. The tradeoffs between decreasing the size of the state space and loss of accuracy must be thoroughly researched before further conclusions can be drawn.

In the evaluation experiment, the assumption of biconditionality between \textit{system outputs} and \textit{system state} (see section~\ref{subsec:stateoutput}) seems to have led to only marginal differences. The systems react to the disturbances at approximately the same time with the same speed. Nonetheless this is but a single comparison. The effects of this assumption \textit{must} be better understood before this extraction approach is used for actual optimization purposes. A number of factors may influence the system behavior because of this simplification. The most dangerous possibility, is that the dynamics of the extracted system can be influenced by the order with which states are discovered. Currently the action set is produced in an increasing order, meaning, that the input values generally increase over time. A consequence of this may be that most states are discovered when the system is reacting to increasing input disturbance, which would lead to a generally less stable system. Simple experiments, such as extractions of the same model using inverted action sets, should offer insights into this problem. Additionally the effect of this biconditionality assumption may not be as visible in this experiment as it would be with more complex models. Further research is necessary to better understand the consequences of this assumption.

Further valuable insight can be gained from looking at the response distributions' correlation at each epoch (see figures~\ref{bw_val_const_state_corr} and~\ref{bw_val_step_state_corr}). Two interesting observations can be made. Firstly, whilst the average correlation for the first input signal (figure~\ref{bw_val_const_state_corr}) averages around 0.4, the correlation of the response distributions to the second disturbance signal is almost zero. The next paragraph offers an opinion as to what may be the cause of this. Secondly, an interesting effect it noticable. The correlation of the responses jumps to much higher values when the system is reacting to changing inputs (around 0.9 in epochs 2 and 3 for the constant disturbance signal and around 0.7 in epoch 13 for the step input signal). This may point to an interesting effect, whereby more dynamic behavior is extracted more accurately than stable behavior. Experiments with extractions using different time step to epoch conversion rates (section~\ref{subsec:timestepsdecisionepochs}) may offer evidence to support or disprove this hypothesis.

A last observation can be made by looking at figures~\ref{bw_val_const_real_mean} and~\ref{bw_val_step_real_mean}. Although the similarity of the responses' means underscores the value of this approach, the strange diverging behavior in steady-state must be viewed critically. Especially in the case of the second disturbance signal (scaled step), the system's mean output value diverges from the source model's both before \textit{and} after the disturbance. The difference before the disturbance may be caused by a different initial simulation state and the difference after the disturbance is probably caused by an uncertainty propagation. The latter effect should be studied in more detail by analysing steady-state responses of Markov Processes. The former problem concerning different initial simulation steps is hard to solve. The divergence before the disturbance is, in all likelihood, a consequence of the biconditionality assumption, discussed in section~\ref{subsec:stateoutput}. This divergence probably also affected the responses' distribution correlation, analysed in the previous paragraph. The steady-state behavior of a system is one of its defining properties and often the only really important one (short term dynamics can often be ignored because of most physical systems' inherit inertia), meaning that this effect must be studied in more depth and its causes identified and corrected.

From a different point of view, this example shows, that the extraction of Markov Decision Processes from Simulink models does have a large computational cost. The extraction of this SISO model required more than 40 hours. Taking into account that the extraction of a more accurate model requires an order-of-magintude increase of the number of probabilistic simulations, the cost becomes even higher. A usable transformation of the Butterworth filter system may, indeed, take up to 400 hours. More complex models, with a much larger action space (action set cardinality in the hundreds or thousands), will lead to even higher time and performance costs. Parallelization approaches are the only solution here, and should thus be researched in more detail.

Finally it should be considered, that even though a successful POMDP extraction may be possible, explicitly solving POMDPs is generally inctractable \cite{littman96}. Consequentially, the usefulness of an extraction may be limited by the computational cost of solving, even approximately, POMDPs with large state spaces.

It can be concluded, that the extraction approach presented in this thesis is promising. Comparisons of the original and the extracted system show that the underlying dynamic behavior has been transfered into the new system description. Nonetheless, the consequences of the assumptions and simplifications made by this approach (see chapter~\ref{methodology}) must the researched in more depth, especially in the field of modelling where small mistakes may have disastrous consequences. Additionaly further study is required to assess the usabilify and usefulness of the produced Markov Processes.


% The results described in chapter~\ref{results} offer an interesting perspective on the usefulness of this approach for the transformation of Simulink models into Markov Decision Processes. This short discussion aims to provide some opinions as to the reasons for some of the positive and negative results seen in chapter~\ref{results}. The opinions here are based on experience with other extraction models and knowledge of certain implementation difficulties. They are not based on facts and may, in fact, be wrong. Nonetheless they try to serve as a basis for potential improvements of the extraction algorithm.

% The approach described in the last few chapters does, indeed, seem to have some merit. Although the MDP's simulated responses have a strong variance and the correlation between distributions is extremely low, the responses do indicate that a certain behavioural transfer has occured. Especially promising are the simulated reponses' means, which definitely indicate similar system reactions for both disturbances.

% The fact that the system reacts similarily even though the disturbances occur at different times indicates that the transition probabilities have not been strongly disturbed by the biconditionality assumption discussed in section~\ref{subsec:stateoutput}. This may however prove to be a false conclusion. The effect that the order of extraction simulations has on the resulting system dynamics must be studied in more depth. This is because the order of discovery may have an effect on the accuracy of the extracted model, because the increasing order of the action set may lead to the discovery of states during an `increasing' system reaction, whilst a decreasing order of the action set may lead to more state discoveries during a `decreasing' system reaction. A simple experiment would entail the comparison of systems extracted from an identical source model, but with inverted action sets.

% The low correlation of distributions in the state space may not be a valuable metric. Reasons for this include the fact that the correlation decreases with an increasingly large state space, even though the underlying system dynamics may remain equally accurate. Additionaly a low state distribution correlation does not necessarily prove that the Simulink and the MDP system behave differently, only, that the distributions over the state space do not correlate \textit{for each epoch}. The distribution correlation for both input signals, but especially the scaled step input, described in sections~\ref{sec:respconst} and~\ref{sec:respstep}, can, in all likelihood, also be increased by increasing the number of probabilistic simulations for each source-state/action tuple (see sections~\ref{sec:probabilisticsimulation} and~\ref{sec:resultsextraction}). The number of probabilistic extractions was set to only 20. A much greater number of simulations is required in order to correctly extract the nature of the random input's distribution. An analysis of the system responses after an extraction with 500 or 1000 probabilistic simulations, is likely to show better results.

% The implementation of this extraction algorithm and the study of its strenghts and weaknesses have shown that, although the approach shows promise, the effect different extraction parameters have on the quality of the extracted model must be studied in more depth before these are used for serious optimization. Nonetheless, it must be remembered, that the only important metric of success is the usefulness of an extracted model for optimization. The advantages of MPD-/POMDP-based optimization may easily outweigh the loss of accuracy an extraction entails. Further study of this approach must take into account the effect of extraction errors on the quality of the final optimal policy.