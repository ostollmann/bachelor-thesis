\chapter*{Abstract}
%Context, Content and Conclusion summarized to 1 page.
% English version:

% Lorem ipsum dolor sit amet, consectetuer adipiscing elit, sed diam nonummy nibh euismod tincidunt ut laoreet dolore magna aliquam erat volutpat. Ut wisi enim ad minim veniam, quis nostrud exerci tation ullamcorper suscipit lobortis nisl ut aliquip ex ea commodo consequat. Duis autem vel eum iriure dolor in hendrerit in vulputate velit esse molestie consequat, vel illum dolore eu feugiat nulla facilisis at vero et accumsan et iusto odio dignissim qui blandit praesent luptatum zzril delenit augue duis dolore te feugait nulla facilisi. Lorem ipsum dolor sit amet, consectetuer adipiscing elit, sed diam nonummy nibh euismod tincidunt ut laoreet dolore magna aliquam erat volutpat. Ut wisi enim ad minim veniam, quis nostrud exerci tation ullamcorper suscipit lobortis nisl ut aliquip ex ea commodo consequat. Duis autem vel eum iriure dolor in hendrerit in vulputate velit esse molestie consequat, vel illum dolore eu feugiat nulla facilisis at vero et accumsan et iusto odio dignissim qui blandit praesent luptatum zzril delenit augue duis dolore te feugait nulla facilisi. Nam liber tempor cum soluta nobis eleifend option congue nihil imperdiet doming id quod mazim placerat facer possim assum.

% This thesis presents a method for extracting Partially Observable Markov Decision Processes from Simulink models. Such extractions are useful for two main reasons. Firstly the modelling of complex non-linear dynamic systems is orders of magnitude easier in a graphical modelling tool such as Simulink. The manual representation of non-trivial dynamic models as Markov Decision Processes is extremely difficult, if not impossible. An extraction tool would thus allow easier modelling whilst still permitting the use of a Markov Decision Process based system representation. The latter is important because Markov Decision Processes have inherit advantages compared to `rule-based' dynamic system respresentation when it comes to optimization. Optimization using Markov Decision Processes is non-myopic, uses decoupled descriptions for the system dynamics and the reward model and can be performed on simple and cheap vector processing units. This thesis presents the theoretical background, approach, implementation and evaluation of such an extraction tool.

This thesis presents a method for extracting Partially Observable Markov Decision Processes from Simulink models. The aim is to combine the modelling power of Simulink with the optimization advantages of POMDPs. An extraction tool allows engineers to model the system in graphical modelling tools such as Simulink, yet also permits non-myopic optimization of the modelled system by transforming it into a POMDP, an otherwise error-prone manual process. This transformation tool was produced as part of a project to develop an intelligent assistant for Waste-to-Energy plant managers. This text presents the theoretical background of dynamic systems, stochastic processes and Markov Decision Processes. It also introduces Simulink, a graphical dynamic system modelling tool. The contribution of this work is the study of and the development of an extraction algorithm, in MATLAB, that permits the transformation of arbitrary Simulink models into POMDPs. A secondary product of this research is the development of a validation tool, also written in MATLAB, necessary to judge the quality of the extracted POMDP. This thesis ends with an evaluation of the extraction approach and algorithm by studying the quality of a POMDP extracted from a Simulink model of a third-order Butterworth filter. The thesis shows that the extraction of Partially Observable Markov Decision Processes from Simulink models is, in general, possible. Nonetheless more research is required before the extracted Markov Processes can be used for optimization in an industrial setting.

%------------------------------------------
%\cleardoublepage
%\chapter*{Zusammenfassung}
%
%Deutsche Version vom Abstract.
%
