\chapter*{Abstract}
%Context, Content and Conclusion summarized to 1 page.
% English version:

% Lorem ipsum dolor sit amet, consectetuer adipiscing elit, sed diam nonummy nibh euismod tincidunt ut laoreet dolore magna aliquam erat volutpat. Ut wisi enim ad minim veniam, quis nostrud exerci tation ullamcorper suscipit lobortis nisl ut aliquip ex ea commodo consequat. Duis autem vel eum iriure dolor in hendrerit in vulputate velit esse molestie consequat, vel illum dolore eu feugiat nulla facilisis at vero et accumsan et iusto odio dignissim qui blandit praesent luptatum zzril delenit augue duis dolore te feugait nulla facilisi. Lorem ipsum dolor sit amet, consectetuer adipiscing elit, sed diam nonummy nibh euismod tincidunt ut laoreet dolore magna aliquam erat volutpat. Ut wisi enim ad minim veniam, quis nostrud exerci tation ullamcorper suscipit lobortis nisl ut aliquip ex ea commodo consequat. Duis autem vel eum iriure dolor in hendrerit in vulputate velit esse molestie consequat, vel illum dolore eu feugiat nulla facilisis at vero et accumsan et iusto odio dignissim qui blandit praesent luptatum zzril delenit augue duis dolore te feugait nulla facilisi. Nam liber tempor cum soluta nobis eleifend option congue nihil imperdiet doming id quod mazim placerat facer possim assum.

This thesis presents a method for extracting Partially Observable Markov Decision Processes from Simulink models. Such extractions are useful for two main reasons. Firstly the modelling of complex non-linear dynamic systems is orders of magnitude easier in a graphical modelling tools such as Simulink. The manual representation of non-trivial dynamic models as Markov Decision Processes is extremely difficult, if not impossible. An extraction tool would thus allow  easier modelling whilst still permitting the use of a Markov Decision Process based system representation. Secondly Markov Decision Processes have inherit advantages compared to `rule-based' dynamic system respresentation when it comes to optimization. Optimization using Markov Decision Processes is non-myopic, is decoupled between system dynamics and the reward model and can be performed on simple and cheap vector processing units, making the use of `smart' controllers possible even in cheap products. This thesis presents the theoretical background, approach and implementation of such an extraction tool.

%------------------------------------------
%\cleardoublepage
%\chapter*{Zusammenfassung}
%
%Deutsche Version vom Abstract.
%
